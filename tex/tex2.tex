\documentclass[dvipdfmx]{jsarticle}

\include{begin}

\section{まとめ}

\subsection{コマンド}

\begin{tabular}{|ll|} \hline
 dir & 現在のディレクトリ(フォルダ)の一覧を表示する \\
 dir \ /w & dir を横並びで表示する \\
 dir \ /p & dir を1画面ごとにページ表示する \\
 cd & 現在のパス(位置)を表示する  \\
 cd \ \verb!<パス>! & \verb!<パス>! の位置に移動する \\
 cd \ .. & 一つ上のディレクトリ(ホーム)に移動する \\
 type \ \verb!<ファイル>! & ファイルの内容を表示する \\ 
 more \ \verb!<ファイル>! & ファイルの内容を1画面ごとに表示する \\
 $>$ & 出力先を画面から(たとえば)ファイルに変える。(例) dir \verb!>! dir.txt \\
 \textbar (バーティカルバー) & コマンドをつなげる。(例) dir | more \\ \hline
\end{tabular}

\subsection{その他}

\begin{tabular}{|l|l|} \hline
 $\uparrow$ $\downarrow$ & ヒストリー機能(過去のコマンド履歴を表示) \\
 \verb![F7]! & 履歴一覧。 $\uparrow$ $\downarrow$ で選択、実行 \\ 
 TABキー & 入力補完 \\
 \%環境変数\%  & 環境変数の内容 \\
 cd \ \%HOME\% & ホームディレクトリ(ホームフォルダ) に移動する \\ \hline
\end{tabular}




\include{end}

%% 修正時刻: Sun Jan  9 07:59:24 2022
