\documentclass[dvipdfmx]{jsarticle}

\include{begin}

\section{バッチファイル}

\subsection{コマンドを作成する}

簡単なコマンドを作成してみる。
以下のコードをテラパッドなどで入力して、``hello.bat''として保存する。
保存先は デスクトップ にしておく。

\begin{lstlisting}[caption=hello.bat]
 @echo off
 echo こんにちは
\end{lstlisting}

現在いる位置はコマンドプロンプトに表示されている。
``C:\yen users\yen USER''である。

そこで、以下のコマンドを入力する。

\fbox{cd \ Desktop} $<$Enterキー$>$

コマンドプロンプトが ``C:\yen users\yen USER\yen Desktop''
となる。

\rightline{
(このとき、\fbox{cd \ De} まで入力したら、\textgt{TABキー} を押すと
\fbox{cd \ Desktop} と補完してくれる。)}

以下のコマンドを入力する。

\fbox{dir} $<$Enter$>$

以下のように、そのフォルダにあるファイルやフォルダが表示される。

\vspace{3mm}
\includegraphics[width=12cm]{img/cmd-04-dir.png}
\vspace{3mm}

その中に ``hello.bat'' があることを確認する。

hello.bat は以下のようにして実行できる。

\fbox{hello}

\vspace{3mm}
\includegraphics[width=15cm]{img/cmd-05-hello.png}
\vspace{3mm}

\subsection{バッチファイル}

今作成した ``hello.bat'' はバッチファイルと呼ばれるもので、
コンピュータに与えるコマンドを手順として多数記述しておいて、
それらを実行させるものである。``スクリプト''とも呼ばれる。

今作成したのは簡単な手順であるが、業務で使われる場合は
複雑なものとなる。

拡張子は ``.bat''である。

``echo off'' は、コマンド文字列を画面に表示させないためのものである。
``@'' は、そのコマンドを画面に表示させないためのものである。




\include{end}

%% 修正時刻: Sat 2022/08/13 10:48:070
