\documentclass[dvipdfmx]{jsarticle}

\include{begin}

\vspace{6mm}
\section{バッチファイル}

\subsection{コマンドを作成する}

簡単なコマンドを作成してみる。
以下のコードをテラパッドなどで入力して、``hello.bat''として保存する。
保存先は デスクトップ にしておく。

\begin{lstlisting}[caption=hello.bat]
 echo こんにちは
\end{lstlisting}

現在いる位置はコマンドプロンプトに表示されている。
``C:\yen users\yen USER''である。

そこで、以下のコマンドを入力する。

\fbox{cd \ Desktop} $<$Enterキー$>$

コマンドプロンプトが ``C:\yen users\yen USER\yen Desktop''
となる。

\rightline{
(このとき、\fbox{cd \ De} まで入力したら、\textgt{TABキー} を押すと
\fbox{cd \ Desktop} と補完してくれる。)}

以下のコマンドを入力する。

\fbox{dir} $<$Enter$>$

以下のように、そのフォルダにあるファイルやフォルダが表示される。

\vspace{3mm}
\includegraphics[width=12cm]{img/cmd-04-dir.png}
\vspace{3mm}

その中に ``hello.bat'' があることを確認する。

hello.bat は以下のようにして実行できる。

$>$ \fbox{hello}

\vspace{3mm}
\includegraphics[width=15cm]{img/hello-01.png}
\vspace{3mm}

画面に \fbox{ echo こんにちは } と表示されてから \fbox{ こんにちは } と
表示されている。
\fbox{ echo こんにちは } が表示されないようにするには、
以下のようにする。

\begin{lstlisting}[caption=hello.bat]
 echo off
 echo こんにちは
\end{lstlisting}

\vspace{3mm}
\includegraphics[width=15cm]{img/hello-02.png}
\vspace{3mm}

\fbox{ echo こんにちは } は消えたけれど、かわりに \fbox{ echo off } が
表示されてしまった。
それも表示されないようにするには、echo off の先頭に "@" をつける。

\begin{lstlisting}[caption=hello.bat]
 @echo off
 echo こんにちは
\end{lstlisting}

\vspace{3mm}
\includegraphics[width=15cm]{img/hello-03.png}
\vspace{3mm}



\subsection{バッチファイル}

今作成した ``hello.bat'' はバッチファイルと呼ばれるもので、
コンピュータに与えるコマンドを手順として多数記述しておいて、
それらを実行させるものである。``スクリプト''とも呼ばれる。

今作成したのは簡単な手順であるが、業務で使われる場合は
複雑なものとなる。

拡張子は ``.bat''である。





\include{end}

%% 修正時刻: Mon Aug 15 06:45:49 2022
