\documentclass[dvipdfmx]{jsarticle}

\include{begin}

\section{コマンドプロンプトとは?}

\subsection{起動}

``スタートボタン'' -- ``Windowsシステムツール'' の中に ``コマンドプロンプト'' はある。

右クリックして、``スタートパネルにピン留めする'' かあるいは、``その他''
-- ``タスクバーにピン留めする'' にしておくとよい。

しかし、ふつうは以下の方法で起動する。

\begin{itemize}
 \item スタートボタン横の検索で ``cmd'' と入力して $<$Enterキー$>$
 \item エクスプローラのURL欄にて ``cmd'' と入力して $<$Enterキー$>$
\end{itemize}

黒いウィンドウが表示される。これがコマンドプロンプト。

\subsection{設定}

ウィンドウ左上のアイコンをクリックすると ``プロパティ''という項目がある。
それを選択すると、設定画面になる。

\subsection{動作を試す}

画面には、以下のような文字が表示されている。

\vspace{3mm}
\includegraphics[width=12cm]{img/cmd-01.png}
\vspace{3mm}


\fbox{\quad C:\\Users\\User$>$ \quad} はプロンプトといい、現在の位置を
ユーザーに示している。
また、$>$ という文字列に続けて文字を入力できることを示している。



試しに以下のコマンドを入力して $<$Enterキー$>$ を押してみる。

\fbox{\quad ver \quad}

\vspace{3mm}
\includegraphics[width=12cm]{img/cmd-02-ver.png}
\vspace{3mm}

Windowsのバージョンが表示される。

今度は以下のコマンドを入力する。($<$Enterキー$>$も)

\fbox{\quad date \quad}

現在の日付が表示されて、次の行で
``新しい日付を入力してください: (年-月-日)``
と表示されて、入力が促される。

何も入力せず、そのまま $<$Enterキー$>$ を押せばよい。

\vspace{3mm}
\includegraphics[width=12cm]{img/cmd-03-date.png}
\vspace{3mm}

\subsection{コマンドプロンプトとは?}

コマンドプロンプトというのは、この黒い画面に文字(コマンド)を入力して
コンピュータからの返答を得るというものである。

つまり、コンピュータとの「対話」処理である。
最近の言葉では「チャット」になる。

コンピュータには二つの種類のアプリケーションがある。

\begin{itemize}
 \item GUIアプリケーション --- マウスで操作するアプリケーション
 \item CUIアプリケーション --- キーボード入力で操作するアプリケーション
\end{itemize}
\begin{flushright}
GUI --- Graphical User Interface \\
CUI --- Character User Interface
\end{flushright}



``WORD''などのアプリは GUIである。大半のアプリが GUI である。

しかし、CUIアプリも数多くある。特にプログラミング言語(PHP、Javaなど)は CUI である。

しかし、CUIだと使いづらいので、Javaでは Eclipse などの統合開発環境(IDE)
を使った開発が行なわれている。

先程の''ver''というのは、ひとつのアプリケーションで、``data''というのも
アプリケーションである。

``help''と入力すれば、コマンドの一覧が表示される。中にはシステムの改変を行うコマンドもあるので、表示されたコマンドを不用意に実行してはいけない。


\subsection{コマンドを作成する}

簡単なコマンドを作成してみる。
以下のコードをテラパッドなどで入力して、``hello.bat''として保存する。
保存先は デスクトップ にしておく。

\begin{lstlisting}[caption=hello.bat]
 @echo off
 echo こんにちは
\end{lstlisting}

現在いる位置はコマンドプロンプトに表示されている。
``C:\yen users\yen USER''である。

そこで、以下のコマンドを入力する。

\fbox{\quad cd \ Desktop \quad} $<$Enterキー$>$

コマンドプロンプトが ``C:\yen users\yen USER\yen Desctop''
となる。

以下のコマンドを入力する。

\fbox{\quad dir \quad} $<$Enter$>$

以下のように、そのフォルダにあるファイルやフォルダが表示される。

\vspace{3mm}
\includegraphics[width=12cm]{img/cmd-04-dir.png}
\vspace{3mm}

その中に ``hello.bat'' があることを確認する。

hello.bat は以下のようにして実行できる。

\fbox{\quad hello \quad}

\vspace{3mm}
\includegraphics[width=15cm]{img/cmd-05-hello.png}
\vspace{3mm}


\subsection{バッチファイル}

今作成した ``hello.bat'' はバッチファイルと呼ばれるもので、
コンピュータに与えるコマンドを手順として多数記述しておいて、
それらを実行させるものである。``スクリプト''とも呼ばれる。

今作成したのは簡単な手順であるが、業務で使われる場合は
複雑なものとなる。

拡張子は ``.bat''である。

``\@echo off'' は、コマンド文字列を表示させないためのものである。


\section{コマンドプロンプトによるディレクトリ(フォルダ)の移動}

コマンドプロンプトには「現在の位置」が表示される。
「現在どの位置にいるのか」を理解する必要がある。

スタート・メニューからコマンドプロンプトを起動した場合、

\fbox{C:\yen Users\yen USER} と表示される。
この場所を \textgt{ホームディレクトリ} あるいは \textgt{ホームフォルダ}
という。

また、この ``C:\yen Users\yen USER'' を \textgt{パス} という。

この場所で \fbox{dir} とすると、この場所にあるファイルやディレクトリ(フォルダ)の一覧が表示される。

\vspace{3mm}
\begin{tabular}{|l|} \hline
Desktop --- デスクトップ \\
Documents --- ドキュメント \\
Downloads --- ダウンロード \\
Music --- ミュージック \\
Pictures --- ピクチャ \\
Videos --- ビデオ \\ \hline
\end{tabular}
\vspace{3mm}

先程作成した ``hello.bat'' を実行するためには デスクトップ に移動する必要がある。\fbox{cd \ desktop} とすると、移動できるし、そこでなら ``hello.bat''は
実行できるが、今は移動せずにこのままでいることにする。

\fbox{dir} とすると、たくさんのファイルやフォルダがあるので、
画面の上に過ぎ去ってしまう。
そこで、以下のようにする。

\fbox{dir \ /p}

すると、1画面分ごとに表示される。
そのとき、一番上には、以下のように表示される。

\vspace{3mm}
\includegraphics[width=10cm]{img/cmd-06-dir.png}
\vspace{3mm}

この \fbox{.} は、「ここ」をあらわす。

\fbox{..} は、「ひとつ上」をあらわす。

だから、``./memo.txt''とすると、このディレクトリ(フォルダ)にある memo.txt
のことになる。(memo.txtが存在するとして)

また、``../some.txt'' とすると、ひとつ上のディレクトリ(フォルダ) にある
some.txt ということになる。(some.txtが存在するとして)

さらに \fbox{cd ..} とすると、ひとつ上のディレクトリ(フォルダ)に移動できる。


\section{システム環境変数の PATH への登録}

デスクトップに hello.bat を作成したが、デスクトップにコマンド置き場として
``myApp''というフォルダを作成し、その中に hello.bat を置くことにする。

\vspace{3mm}
\includegraphics[width=8cm]{img/cmd-07.png}
\vspace{3mm}

この hello.bat を実行しようとすると、現状では
``C:\yen Users\yen USER\yen Desktop\yen myApp'' に移動しなくてはならない。

\fbox{cd \ C:\yen Users\yen USER\yen Desktop\yen myApp}

しかし、Windowsには ``システム環境変数''という仕組みがあり、そこに
``PATH''という変数が用意されていて、その変数に、``myApp'' のフォルダを
登録すると、このコンピュータのどこからでも hello.bat を呼び出すことができる。

\subsection{システム環境変数の編集}

スタートボタンの横の検索に ``システム環境変数'' と入力すると、
``システム環境変数の編集''という文字が現れるので、それをクリックする。

\vspace{3mm}
\includegraphics[width=8cm]{img/cmd-08.png}
\vspace{3mm}

開いたウィンドウで、``環境変数'' をクリックする。

環境変数のダイアログが開くので、``システム環境変数''の ``Path'' を
選択し、``編集''をクリックする。

\vspace{3mm}
\includegraphics[width=8cm]{img/cmd-09.png}
\vspace{3mm}


環境変数名の編集ダイアログが開くので、``新規''を選択する。

\vspace{3mm}
\includegraphics[width=8cm]{img/cmd-10.png}
\vspace{3mm}

入力欄ができる。``参照''をクリックする。

\vspace{3mm}
\includegraphics[width=8cm]{img/cmd-11.png}
\vspace{3mm}

開いたダイアログで デスクトップの ``myApp'' を選択して ``OK''。

\vspace{3mm}
\includegraphics[width=8cm]{img/cmd-12.png}
\vspace{3mm}

``C:\yen Users\yen USER\yen Desktop\yen myApp'' が環境変数''Path'' に
登録された。

\vspace{3mm}
\includegraphics[width=8cm]{img/cmd-13.png}
\vspace{3mm}

あとは、``OK''をクリックしてダイアログを閉じていく。
``X''(閉じる)をクリックすると、今までの操作がすべてキャンセルされるので
気をつける。

このように システム環境変数の``Path'' にアプリのある場所を登録することで、
その場所にいなくても、そのアプリを実行できるようになる。

ただ、現在開いているコマンドプロンプトはいったん閉じて、
再度開きなおさないとこの変更は反映されない。


\vspace{3mm}
\includegraphics[width=12cm]{img/cmd-14-hello.png}
\vspace{3mm}

\newpage















> netstat -noa

ポート番号とPIDがわかる

> netstat -nba

ポート番号とそれを使っているアプリがわかる



> ver

Windowsのバージョン


> ipconfig

現在のネットワーク(IP)構成


> nslookup google.co.jp

IPアドレスを調べる


> dir

現在のフォルダにあるファイルとフォルダの一覧

> dir /w

横に広げて表示


> cd

現在のフォルダを表示


> cd %HOME%

ホームフォルダへ移動






\include{end}

%% 修正時刻: Mon Jan  3 22:12:08 2022
